\chapter*{Avant-propos}

Toutes les cellules d’un organisme multicellulaire partagent, à de rares exceptions près, exactement la même information génétique, les mêmes molécules d’ADN, le même répertoire de gènes. Et pourtant chacune de ces cellules, que ce soit une cellule musculaire ou un neurone, est spécialisée pour assurer des fonctions qui lui sont propres, produisant alors une diversité et une quantité de protéines bien à elle. La production variable des différentes protéines, à partir d’un même génome est permise grâce à des processus activant et désactivant spécifiquement certains gènes. Au cours de la différenciation et de la spécialisation cellulaire, de nombreux mécanismes moléculaires permettent la régulation de l’expression des gènes. L’identification des mécanismes de régulation de l’expression est encore aujourd’hui un vrai défi tant ils sont multiples et variés. Depuis le séquençage du génome humain en 2001, de nombreuses analyses ont tenté de décrypter ses fonctions \citep{venter_sequence_2001}. Seulement 2\% des bases du génome codent pour des protéines, mais de nombreux autres éléments fonctionnels sont présents et pourraient avoir des rôles importants dans le contrôle de l'expression des gènes. Leur identification et leur caractérisation sont cruciales pour comprendre le fonctionnement du génome, de la cellule et plus largement des variations entre les organismes et les espèces. \\

L'étude des mécanismes régulateurs de l’expression des gènes et leur évolution pourrait aider à mieux comprendre l’évolution morphologique, physiologique et écologique du vivant. Les changements des séquences des protéines sont les mécanismes les plus généralement avancés pour expliquer les différences observables, dites phénotypiques, entre individus et entre espèces. Cependant, les espèces de vertébrés notamment, montrent d’importantes variations phénotypiques alors qu’elles partagent une grande partie de leurs gènes codants pour des protéines qui ont en plus des séquences très proches \citep{ponting_functional_2008}. Dès 1975, l’idée que la divergence génétique des séquences codantes entre l’homme et le chimpanzé ne suffisait pas pour expliquer les différences phénotypiques avait été proposée \citep{king_evolution_1975}. Des modifications d’expression des gènes \textit{via} des changements au niveau des patrons de régulation seraient alors en mesure d’expliquer une grande partie des différences entre les espèces, et ainsi constituer une part importante de la base génétique de l’évolution. \\

La compréhension des relations entre les modifications des procédés de régulation, les variations de l’expression des gènes et les variations phénotypiques entre espèces est ainsi au centre de mon travail détaillé dans ce manuscrit. 
