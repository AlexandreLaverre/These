\chapter*{Résumé}
\label{résumé}

Chez les organismes multicellulaires, la régulation de l'activité des gènes est nécessaire pour établir des profils d’expression spécifiques garantissant le développement et la diversité fonctionnelle de l’organisme. Des variations de la régulation des gènes pourraient expliquer une partie importante des différences observées entre les espèces, en permettant d’affiner l’activité des gènes sans modifier leur séquence. L’expression des gènes est régulée par des interactions moléculaires complexes, impliquant notamment des éléments régulateurs agissant en \textit{cis}, c’est-à-dire se trouvant sur le même chromosome que leurs gènes cibles. L’identification de ces éléments dans les génomes est désormais possible grâce au développement de nombreuses techniques biomoléculaires. Cependant, définir les associations entre les gènes et les éléments \textit{cis}-régulateurs reste un défi, d’autant plus que ces derniers peuvent se situer à grande distance génomique des gènes qu’ils contrôlent. Des boucles de chromatine mettent en interaction physique les éléments \textit{cis}-régulateurs distaux et les promoteurs des gènes. Il est aujourd’hui possible de détecter ces boucles avec des méthodes de capture de la conformation de la chromatine. De données s’accumulent et permettent de mieux définir les paysages \textit{cis}-régulateurs et de comprendre les implications des contacts de chromatine dans l’expression des gènes et son évolution.\\

Durant cette thèse, j’ai d’abord combiné et analysé des cartes de contacts de la chromatine obtenues pour plusieurs types cellulaires chez la souris et chez l’humain. J’ai comparé les paysages \textit{cis}-régulateurs définis par les approches classiques de voisinage et par les contacts de la chromatine en développant GOntact. Cet outil permet d’inférer un enrichissement fonctionnel pour un ensemble d’éléments régulateurs, à partir de l'annotation fonctionnelle des gènes contactés par ces éléments. En utilisant ces données de contacts, j’ai ensuite montré qu'il existe une conservation évolutive significative des paysages \textit{cis}-régulateurs chez les vertébrés, indiquant que la sélection naturelle agit pour préserver non seulement les séquences des éléments régulateurs mais aussi leurs contacts chromatiniens avec les gènes cibles. Nous avons montré que l'évolution des paysages \textit{cis}-régulateurs, mesurée en termes de séquences des éléments, de synténie ou de contacts avec des gènes cibles, est significativement associée à l'évolution de l'expression des gènes. Finalement, j’ai cherché à évaluer la relation entre l’évolution des éléments \textit{cis}-régulateurs et l’évolution phénotypique des organismes. J’ai étudié la perte convergente du phallus intervenue chez plusieurs lignées d’oiseaux. Le déterminisme génétique de ce changement phénotypique est encore inconnu mais semble impliquer des modifications importantes des patrons d’expression de gènes du développement, dont la séquence est très conservée. J'ai réalisé une étude comparative sur les génomes complets d'oiseaux, en utilisant des données moléculaires mesurant l'expression des gènes et détectant l’ouverture de la chromatine. En particulier, j’ai analysé l’évolution des séquences potentiellement régulatrices. Ceci nous a permis de proposer des régions candidates associées à ce changement phénotypique. \\

Les résultats obtenus au cours de cette thèse ont permis de confirmer l’importance de l’utilisation des contacts de la chromatine pour associer les promoteurs des gènes à leurs éléments régulateurs dans un contexte évolutif. En redéfinissant les paysages \textit{cis}-régulateurs, nous avons montré que leur évolution permet d’expliquer l’évolution de l’expression des gènes et que des nouvelles fonctions peuvent être attribuées aux éléments régulateurs. Nos résultats confirment également que l'étude de l'évolution de ces éléments \textit{cis}-régulateurs peut fournir des explications à des changements morphologiques majeurs.\\

\chapter*{Abstract}
\label{abstract}
In multicellular organisms, gene expression is regulated by complex molecular interactions involving \gls{cis}-acting regulatory sequences, which are located on the same chromosome as their target genes, and \gls{trans}-acting factors present in nucleus. Identifying these elements is now possible thanks to the development of numerous molecular techniques. However, defining the associations between genes and \gls{cis}-regulatory sequences remains challenging, not least because they may be located at large genomic distances from the genes they control. These distal \gls{cis}-regulatory elements and their target gene promoters are brought into physical proximity in the nucleus through chromatin loops. These loops can now be detected through chromosome conformation capture ("C") techniques. Such data are now rapidly accumulating, allowing us to better define \gls{cis}-regulatory landscapes and to understand the implications of chromatin contacts in gene expression and its evolution.\\

During my PhD, I combined and analysed chromatin contact maps obtained with the Promoter Capture Hi-C (PCHi-C) technique for multiple cell types in mouse and human. First, I compared the \gls{cis}-regulatory landscapes defined by classical genomic approaches and by chromatin contacts. I developed GOntact, a tool that infers functional enrichments for sets of regulatory elements, based on the annotations of the genes they contact. Second, using these chromatin contact data, I showed that there is significant evolutionary conservation of \gls{cis}-regulatory landscapes in vertebrates, indicating that natural selection acts to preserve not only the sequences of regulatory elements but also their chromatin contacts with target genes. I showed that the evolution of \gls{cis}-regulatory landscapes, measured in terms of element sequences, synteny or chromatin contacts with target genes, is significantly associated with the evolution of gene expression.Finally, I sought to assess the relationship between the evolution of \gls{cis}-regulatory elements and the phenotypic evolution of organisms. I studied the convergent loss of the phallus in several bird lineages. The genetic determinism of this phenotypic change is still unknown but seems to involve changes in the expression patterns of highly conserved developmental genes. Using open chromatin data and comparative genomic approaches across several bird lineages, I analyzed the evolution of potential \gls{cis}-regulatory sequences. This allowed us to propose candidate regions involved in the development of the phallus in birds.\\

The results obtained in this thesis highlight the importance of chromatin contacts in determining the associations of genes with their regulatory elements in an evolutionary context. By redefining \gls{cis}-regulatory landscapes, we showed that their evolution can explain the evolution of gene expression and that new functions can be attributed to regulatory elements. Our results also confirm that studying the evolution of these \gls{cis}-regulatory elements can provide hypotheses for major morphological changes.
