\chapter{Discussion}
{\hypersetup{linkcolor=GREYDARK}\minitoc}
\label{chap:discussion}

\section{Estimation des paysages cis-régulateurs de l’expression des gènes grâce aux contacts de chromatine}

\subsection{Un nouveau regard sur les paysages cis-régulateurs}
\subsubsection*{Une mesure expérimentale de l’organisation spatiale du génome}
\subsubsection*{Des paysages cis-régulateurs cohérents avec les fonctions des gènes}

\subsection{Limitations des contacts de chromatine}
\subsubsection*{Des données rares et complexes}
\subsubsection*{Des contacts de chromatine ignorés}
\subsubsection*{La régulation en absence de boucle promoteur-enhancer}

\subsection{Vers une multiplication des approches d’association}
\subsubsection*{Complémentarité des approches de voisinage et de contact dans GOntact}
\subsubsection*{Nombreuses méthodes d'inférence de relations régulatrices}

\section{Corrélation entre l’évolution de l’expression des gènes et des paysages cis-régulateurs}

\subsection{Contacts de chromatine conservés entre humain et souris}
\subsection{Absence de corrélation avec l’évolution du niveau d’expression des gènes}
\subsection{Corrélation avec l’évolution du patron d’expression des gènes}
\subsection{Analyse d’autres catégories de gènes}

\section{Relation avec l’évolution phénotypique}
\subsection{Patrons d’expression conservés malgré les changements phénotypiques}
\subsection{Évolution rapide des paysages cis-régulateurs}
\subsection{Pluralité des mécanismes génomiques sous-jacents}
