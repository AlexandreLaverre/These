\chapter{Discussion}
{\hypersetup{linkcolor=GREYDARK}\minitoc}
\label{chap:discussion}

\section{Estimation des paysages \textit{cis}-régulateurs de l’expression des gènes grâce aux contacts de chromatine}

Le premier objectif de ma thèse a été de définir les paysages \gls{cis}-régulateurs de l'expression des gènes à l'aide de données de conformation de la chromatine chez l'humain et la souris. Celles-ci permettent d'associer les gènes à leurs éléments \gls{cis}-régulateurs d'une manière expérimentale qu'il était important de comparer avec les approches communément employées. 

\subsection{Un nouveau regard sur les paysages \textit{cis}-régulateurs}
\subsubsection*{Une mesure expérimentale de l’organisation spatiale du génome}

Les données de contacts de chromatine centrés sur les promoteurs (\acrshort{PCHi-C}) permettent une prédiction expérimentale des paysages \gls{cis}-régulateurs qui est indépendante de l’organisation des gènes. La grande majorité des autres approches d’association entre les gènes et les éléments \gls{cis}-régulateurs se basent sur des inférences strictement computationnelles à partir du voisinage immédiat (en termes de séquence génomique linéaire) autour du promoteur ou du site d’initiation de la transcription du gène. Dans les approches par voisinage que j’ai mentionnées au cours de cette thèse, le domaine de régulation défini pour un gène est directement dépendant de la distance qui le sépare des gènes voisins. La distribution des gènes dans le génome n’étant pas homogène, la densité locale en gènes est alors un facteur déterminant dans la définition des domaines de régulation et donc dans l’attribution des éléments \gls{cis}-régulateurs à leurs gènes cibles. Par exemple, les gènes séparés de leurs plus proches voisins par de grandes distances génomiques se verront attribuer un grand nombre d’éléments \gls{cis}-régulateurs. Avec une telle définition, les frontières des domaines de régulation sont également dépendantes de la qualité des annotations des gènes. En retirant l’annotation d’un gène, la taille des domaines de régulation des gènes avoisinants est artificiellement augmentée, ce qui modifie l’attribution des éléments \gls{cis}-régulateurs aux gènes. Lorsque la distance entre gènes voisins est définie comme la distance entre leurs promoteurs, comme dans GREAT \citep{mclean_great_2010}, la détermination correcte des sites d’initiation de la transcription joue également un rôle. Dans l’outil GREAT par exemple, seuls les gènes possédant une annotation d’ontologie sont utilisés pour définir les domaines de régulation \citep{mclean_great_2010}. Les paysages \gls{cis}-régulateurs sont dans ces cas directement influencés par l'organisation du génome, par la qualité des annotations génomiques et par les critères de sélection des gènes à analyser. \\

Au cours de mes travaux, j’ai pu confirmer que les éléments \gls{cis}-régulateurs qui contactent physiquement les gènes ne sont majoritairement pas ceux qui sont les plus proches sur la séquence d’ADN \citep{smemo_obesity-associated_2014}. De plus, il n’existe qu’une faible concordance entre les deux approches pour ce qui est du nombre d’éléments \gls{cis}-régulateurs attribués aux gènes. Les données de contact de chromatine révèlent des interactions à plusieurs centaines de milliers de paires de bases \citep{laverre_long-range_2022}. Ces relations à grande distance sont pour la majorité exclues \textit{a priori} par des méthodes d’inférence par voisinage. Seuls certains gènes séparés de leurs voisins les plus proches par de très grandes distances intergéniques présentent des associations à longue distance partagées par les deux méthodes (Partie \ref{part:chap2}). A courte distance une plus grande proportion de paires entre gènes et éléments \gls{cis}-régulateurs sont communes. D’autres approches d'inférence des relations régulatrices donnent également beaucoup de poids au voisinage, comme une méthode récente d’association par co-activité des gènes et des \glspl{amplificateur} \citep{hait_ct-focs_2022}. Dans celle-ci, seuls les N \glspl{amplificateur} les plus proches des gènes, où N est un nombre fixé à l'avance, sont considérés pour prédire les relations régulatrices. Celà permet de s’affranchir de la distribution hétérogène des gènes dans le génome mais la structure des paysages \gls{cis}-régulateurs inférés dépend alors de la distribution des amplificateurs.\\

Les données de \acrshort{PCHi-C} ne sont toutefois pas entièrement indépendantes de l’organisation génomique. La taille des régions contactées est dépendante de la distribution des sites de reconnaissance des enzymes de restriction utilisées dans les protocoles de capture de conformation de la chromatine. En effet, la fragmentation du génome au niveau de motifs identiques est nécessaire afin que les régions en proximité physique puissent se liguer deux-à-deux par leurs extrémités. Les sites de fixation de l’enzyme HindIII, employée pour produire les données de \acrshort{PCHi-C} avec lesquelles j’ai travaillé, sont nombreux mais distribués de manière hétérogène dans les génomes de l’humain et de la souris. La taille des régions en contact n’a donc pas de sens biologique et peut être variable. De plus, la taille de ces régions est positivement corrélée avec la probabilité qu'ils recouvrent des promoteurs ou des éléments \gls{cis}-regulateurs. Il est ainsi possible d’obtenir un grand nombre de paires promoteur - \gls{amplificateur} avec un seul contact entre deux régions, sans savoir laquelle ou lesquelles sont biologiquement pertinentes. 

\subsubsection*{Des paysages \textit{cis}-régulateurs cohérents avec les fonctions des gènes}

Pour aller plus loin dans la description des paysages \gls{cis}-régulateurs identifiés par \acrshort{PCHi-C} nous avons developpé GOntact, un outil qui permet d’inférer les fonctions d'un ensemble d'éléments \gls{cis}-régulateurs à partir des annotations fonctionnelles des gènes qu’ils contactent (par exemple celles fournies par Gene Ontology). Le principe de GOntact est similaire à celui de GREAT, qui est couramment utilisé pour annoter les séquences potentiellement \gls{cis}-régulatrices selon les fonctions des gènes voisins et qui a largement fait ses preuves dans de nombreux contextes. En utilisant des éléments \gls{cis}-régulateurs validés expérimentalement et actifs au cours du développement embryonnaire dans le coeur et le cerveau, les enrichissements ontologiques ont révélé des fonctions métaboliques pertinentes avec les deux approches. Malgré les différences d’attribution des éléments à leurs gènes cibles, certaines fonctions sont retrouvées comme enrichies communément par GREAT et par GOntact. D'autres fonctions pertinentes sont détectées uniquement dans l'une des deux approches. La combinaison des méthodes pourraient ainsi apporter des informations complémentaires pour mieux comprendre les rôles des éléments \gls{cis}-régulateurs.

\subsubsection*{Des paysages \textit{cis}-régulateurs cohérents avec l'expression des gènes}
Nous avons confirmé que les contacts de chromatine sont enrichis en relations potentiellement régulatrices \citep{mifsud_mapping_2015, schoenfelder_pluripotent_2015, javierre_lineage-specific_2016}. Les fragments de restriction contactés par les promoteurs des gènes contiennent plus d’amplificateurs, prédits par diverses sources, que la moyenne du génome. Les analyses que j’ai présentées ont également confirmé que les caractéristiques des paysages \gls{cis}-régulateurs déterminés par \acrshort{PCHi-C} sont cohérents avec l’expression des gènes. \\

Premièrement, la complexité des paysages \gls{cis}-régulateurs en terme de nombres d’\glspl{amplificateur} présents dans les régions contactées, est corrélée positivement au niveau d’expression moyen des gènes au sein d’un type cellulaire. Ceci est cohérent avec les modèles de régulation où la fréquence de transcription d’un gène dépend de l’interaction avec plusieurs \glspl{amplificateur} actifs dont l’effet est additif \citep{mifsud_mapping_2015, schoenfelder_pluripotent_2015, javierre_lineage-specific_2016}. Plus ces derniers sont nombreux dans l’environnement physique d’un gène et plus la probabilité d’interaction et donc de transcription est élevée. Il a déjà été montré que les régions contactées par les promoteurs sont enrichies en marques d’histones associés à des \glspl{amplificateur} actifs \citep{javierre_lineage-specific_2016}. La colocalisation des gènes et des éléments \gls{cis}-régulateurs actifs dans des zones du noyau où la transcription est importante pourrait contribuer à cette observation \citep{sutherland_transcription_2009}. Avec une approche de voisinage, en considérant l’ensemble des éléments \gls{cis}-régulateurs répertoriés par le consortium ENCODE, tous tissus et types cellulaires confondus, nous n’avons observé aucune corrélation entre la complexité du paysage et le niveau d’expression (Partie \ref{part:chap2}; \citet{laverre_long-range_2022}). Ceci est contraire à ce qui a été montré en analysant uniquement les éléments éléments \gls{cis}-régulateurs actifs dans le même type cellulaire que celui où on analyse les niveaux d'expression \citep{berthelot_complexity_2018, naville_long-range_2015}. Dans ces études, le nombre d’éléments \gls{cis}-régulateurs actifs dans le voisinage d’un gène est corrélé à son niveau d’expression. Ces résultats suggèrent qu'une part importante de la régulation de l'expression des gènes s'effectue par des éléments \gls{cis}-regulateurs voisins. Ces observations ne sont pas incompatibles, l’entourage physique d’un gène est à la fois composé des éléments voisins sur la séquence mais également d’éléments plus distants rapprochés par des contacts de chromatine. Ceci révèle néanmoins une différence fondamentale entre mes analyses à partir de données combinées de prédictions d’éléments \gls{cis}-régulateurs sur plusieurs \glspl{condition} (tissus, types cellulaires, stades de développement...) et les études à l’échelle d’une seule \gls{condition}. Dans nos définitions de paysage \gls{cis}-régulateur, selon la méthode d’association, les éléments \gls{cis}-régulateurs sont attribués aux gènes sans tenir compte de leur patrons d'activité. Autrement dit, la co-activité entre élément \gls{cis}-régulateur et gène n’est pas prise en compte. Le nombre de relations régulatrices entre gène et \glspl{amplificateur} est donc nécessairement surévalué. Notre objectif était de considérer l’ensemble des éléments \gls{cis}-régulateurs potentiels d’un gène. L’absence de corrélation entre le nombre d’éléments \gls{cis}-régulateurs voisins d’un gène et son niveau d’expression pourrait ainsi indiquer qu’une part importante de ceux-ci ne participe pas à son expression ou du moins qu’ils ne le font pas simultanément dans le même tissu. Pour les paysages \gls{cis}-régulateur mesurés par \acrshort{PCHi-C}, en identifiant les \glspl{amplificateur} actifs, par l’ouverture de la chromatine ou des modifications d’histones par exemple, on pourrait s’attendre a observer une relation avec le niveau d’expression encore plus importante. \\

L’analyse conjointe des données de \acrshort{PCHi-C} et des patrons de l’expression des gènes sur plusieurs échantillons nous a permis de montrer que la complexité des paysages \gls{cis}-régulateurs est positivement corrélée à l’étendue de l’expression des gènes. Plus le nombre d’éléments \gls{cis}-régulateurs contactés par un gène est elevé, plus celui-ci est exprimé dans un grand nombre de \glspl{condition}. Ceci indique qu’indépendamment de l’activité des éléments \gls{cis}-régulateurs, les contacts de chromatine peuvent donner des informations sur le patron d’expression des gènes. Nous avons ainsi confirmé que de nombreux contacts entre un gène et des éléments \gls{cis}-régulateurs sont observés en l’absence d’expression comme le cas du gène \textit{SHH} et de ZRS \citep{schoenfelder_pluripotent_2015}. Ces contacts pré-formés pourraient faciliter l'activation de gènes cruciaux pour l’organisme ou dont l’expression nécessitent une adaptation rapide à des facteurs environnementaux \citep{ing-simmons_independence_2021}. Dès l’activation d’un amplificateur, les gènes qui le contactent pourraient être transcrits sans modification majeure de la structure de la chromatine. La conformation spatiale partiellement pré-formée des génomes permet donc aussi d’inférer des paysages \gls{cis}-régulateurs indépendamment de l’activité des gènes. L’accumulation d’information des contacts de chromatine pourraient ainsi révéler le paysage \gls{cis}-régulateur plus complet d’un gène qui n’est pas nécessairement spécifique d’un type cellulaire. \\

On peut se questionner sur la causalité entre la complexité des paysages \gls{cis}-régulateurs et l’étendue de l’expression des gènes. Il a déjà été montré qu’un gain d’\gls{amplificateur} peut participer à activer l’expression d’un gène dans un nouveau contexte \citep{rebeiz_evolutionary_2011, thompson_novel_2018}. Mais le nombre d’éléments \gls{cis}-régulateurs d’un gène pourrait également augmenter en conséquence de l’étendue ou de la pléiotropie des gènes \citep{monteiro_identifying_2016}. Selon un modèle de Duplication Dégénération et Complémentation, un \gls{amplificateur} \gls{pleiotrope} ancestral pourrait se dupliquer en plusieurs copies qui évolueraient vers une sous-fonctionnalisation. Cette modularité de différents éléments \gls{cis}-régulateurs paralogues pourrait alors affiner la régulation de l’expression des gènes cibles dans plusieurs contextes \citep{murugesan_evolution_2022}. Des analyses comparatives de la fonction et de l’évolution des éléments \gls{cis}-régulateurs sont nécessaires pour mieux comprendre l’origine de la complexité des paysages \gls{cis}-régulateurs et sa relation avec l’évolution de l’étendue d’expression des gènes.

\subsection{Limitations des contacts de chromatine}
\subsubsection*{Des données rares et complexes}
Les données de conformation de la chromatine comme le \acrshort{PCHi-C} sont encore difficilement accessibles. Ces données sont produites grâce à des techniques bio-moléculaires complexes et coûteuses et dont les protocoles peuvent être difficiles à mettre en place. Les sondes génétiques qui permettent de cibler les promoteurs des gènes sont notamment dépendantes du génome considéré et sont actuellement disponibles uniquement pour l’humain et la souris, ce qui complique l’emploi sur des espèces non modèles. Ces sondes spécifiques ne ciblent d’ailleurs pas tous les gènes (70\% des gènes codants pour des protéines chez l’humain), ce qui ne permet donc pas d’étudier l’ensemble des paysages \gls{cis}-régulateurs. La non-homogénéité des protocoles, notamment par l’utilisation d'enzymes de restriction différentes, peut également compliquer les comparaisons entre les études. De plus, le traitement bioinformatique et statistique de ces données demande du temps et d’importantes ressources. Durant ma thèse, j’ai par exemple compilé toutes les données disponibles de \acrshort{PCHi-C} aux protocoles identiques, ce qui représente plus de 4 Téraoctets de lectures de séquençage brutes et, une fois le traitement bio-informatique maîtrisé, plus d’1 mois de traitement sur le serveur de calcul du LBBE. \\

Les méthodes de Hi-C, qui mesurent les contacts de chromatine sur l’ensemble des loci, commencent à s’accumuler et à se populariser notamment pour l’amélioration de l’assemblage de génome \citep{ghurye_integrating_2019}. À l’origine, la résolution de cette technique ne permettait pas d’inférer avec précision les paysages \gls{cis}-régulateurs des gènes. L’augmentation de la profondeur de séquençage et le développement de méthodes statistiques récentes pourraient cependant permettre d’améliorer sa précision et sa sensibilité \citep{lagler_hic-act_2021}. Les nombreuses données publiées de Hi-C pourraient être exploitées pour étudier les paysages \gls{cis}-régulateurs dans un plus grand nombre de tissus et d’espèces. Par exemple, dans l’étude de l’évolution du phallus chez les oiseaux, nous comptons prochainement traiter des données récentes de Hi-C disponibles pour le canard et le poulet pour améliorer les prédictions d’associations entre les gènes et les éléments \gls{cis}-régulateurs \citep{zhu_three_2021, fishman_3d_2019, li_comparative_2022}.

\subsubsection*{Des contacts de chromatine ignorés}
Dans nos études, nous avons analysé uniquement les contacts intra-chromosomiques entre régions promotrices et non-promotrices. Ces interactions en \gls{cis} sont les plus fréquentes et décrivent des relations régulatrices les plus généralement admises. Cependant plus de 10\% des contacts étudiés chez l’humain mettent en interaction deux régions promotrices. Les premières études ayant analysé des données de contact ont montré que ces interactions sont enrichies en gènes fonctionnant dans des voies métaboliques proches. Les gènes mis en contact ont tendance à fixer les mêmes facteurs de transcription ce qui suggère qu'il pourrait s'agir de gènes co-régulés \citep{schoenfelder_pluripotent_2015}. Ces gènes pourraient être co-localisés dans des micro-environnements du noyau pour coordonner la régulation de leur expression \citep{osborne_active_2004}. Ces micro-environnements sont communément appelés des “usines” de transcription. Dans ces “usines”, la concentration en facteurs de transcription, en ARN polymérase active et en molécules associées aux modifications des ARN messagers est plus élevée \citep{sutherland_transcription_2009, rieder_transcription_2012}. Les régions promotrices ciblées pouvant être relativement grandes, il est également probable qu’elles contiennent d’autres éléments \gls{cis}-régulateurs en plus des promoteurs. Ces contacts entre promoteurs pourraient donc agir d’une manière similaire aux contacts régulateurs “classiques”, qui ont lieu entre promoteurs et amplificateurs. De plus, les fonctions des promoteurs et des \glspl{amplificateur} peuvent être très similaires. Ils peuvent par exemple fixer des facteurs de transcription communs. Il a été montré que 2 à 3\% des promoteurs chez l’humain présentent une activité d’\gls{amplificateur} de l’expression de gènes distaux \citep{dao_genome-wide_2017}. Cette observation pourrait venir complexifier la caractérisation des paysages \gls{cis}-régulateurs.\\

Des contacts inter-chromosomiques sont également présents dans les données de \acrshort{PCHi-C}, ils représentent 1,5\% des interactions combinées chez l’humain. De tels contacts ont déjà été observés pour des gènes activement transcrits et révèlent des relations régulatrices fonctionnelles \citep{spilianakis_interchromosomal_2005}. Ces interactions en \gls{trans} questionnent les mécanismes moléculaires à l’origine de l’organisation spatiale des chromosomes dans le noyau et le maintien de ces loci en proximité physique sans boucle de la chromatine classique. Le facteur de transcription CTCF pourrait jouer un rôle d’intermédiaire pour diriger ces régions en proximité physique \citep{ling_ctcf_2006}. Des boucles de chromatine en \gls{cis} de différents chromosomes pourraient également se colocaliser dans les mêmes usines de transcription. Les contacts interchromosomiques observés seraient alors le résultat d’interactions indirectes via les usines de transcription. L’analyse de la co-expression de ces gènes, ainsi que l’observation en microscopie optique de leur localisation dans le noyau pourrait permettre de mieux comprendre ces interactions atypiques.

\subsubsection*{La régulation en absence de boucle promoteur-enhancer}
Bien que le modèle de contact physique entre promoteur et \gls{amplificateur} soit largement supporté par de nombreuses données et études distinctes, plusieurs observations indiquent que les boucles de la chromatine ne sont pas obligatoires pour les interactions régulatrices. Certaines relations essentielles entre \glspl{amplificateur} et promoteurs existent sans contact physique classique apparent \citep{alexander_live-cell_2019}. Par exemple, le gène \textit{Shh} s’éloigne physiquement de plusieurs de ses \glspl{amplificateur} essentiels dans le développement du tube neural chez la souris alors que son expression augmente \citep{benabdallah_decreased_2019}. Plusieurs modèles ont été proposés pour expliquer de telles relations régulatrices, comme le “kiss and run model”, où le contact entre un \gls{amplificateur} et le promoteur d’un gène permet de transférer des facteurs de transcription mais n’a pas besoin d’être maintenu en proximité pour être fonctionnel ou encore le modèle d’un gradient d’activité de facteur de transcription \citep{karr_transcription_2022}. \\

Des études expérimentales ont également montrées qu’en retirant artificiellement les principales molécules responsables de l’organisation tridimensionnelle de la chromatine, comme le facteur de transcription CTCF ou la cohésine, la plupart des gènes continue à être exprimés à des niveaux normaux \citep{schwarzer_two_2017, rao_cohesin_2017}. Dans ces expériences, les domaines d’association topologique (TAD) et la majorité des boucles de chromatine à grande distance sont éliminées. Cette délétion entraîne peu de changement sur les gènes normalement inactifs dans ces échantillons, mais modifie l’expression d’une minorité des gènes déjà exprimés. La condensation de la chromatine et les marques des histones sont conservées et pourraient expliquer cette observation. Ces marques épigénétiques définiraient des compartiments génomiques à fine échelle indépendamment de la conformation tridimensionnelle des chromosomes \citep{schwarzer_two_2017}. Parmi les gènes dont l’expression est modifiée, une plus forte proportion de gènes subissent une diminution qu’une augmentation de l’expression. De plus, les gènes entourés par de grandes régions intergéniques sont les plus impactés par cette diminution \citep{schwarzer_two_2017}. Ceci pourrait donc s’expliquer par la perte de contact avec des éléments \gls{cis}-régulateurs distaux. En l’absence de cohésine, les gènes avec de nombreux éléments \gls{cis}-régulateurs proches voisins ont tendance à être surexprimés par rapport aux gènes qui en ont moins. Une plus forte compartimentalisation du génome en structures plus restreintes est en effet observée dans ces expériences \citep{rao_cohesin_2017}. En l’absence de contact à longue distance, les éléments \gls{cis}-régulateurs pourraient ainsi être redirigés vers les gènes à proximité. Cette réorganisation des paysages \gls{cis}-régulateurs et la possible redondance des séquences \gls{cis}-régulatrices pourraient permettre de temporiser des variations de structure mais pourrait avoir des conséquences plus grandes à large échelle sur le patron d’expression. La caractérisation plus fine des paysages \gls{cis}-régulateurs des gènes à l’aide de données à plus haute résolution comme le \acrshort{PCHi-C} pourrait permettre de mieux comprendre la dynamique des boucles de chromatine. Plus généralement, ces études confirment que d’autres mécanismes que les contacts à grande distance sont à l’oeuvre dans la mise en place des associations spécifiques entre les promoteurs des gènes et les éléments \gls{cis}-régulateurs (cf modèle complet de \citet{schoenfelder_long-range_2019}).

\subsection{Vers une multiplication des approches d’association}
\subsubsection*{Complémentarité des approches de voisinage et de contact dans GOntact}

Les approches de voisinage sont dépendantes de l’organisation génomique pour définir les associations entre gènes et éléments \gls{cis}-régulateurs et sont par définition limitées aux régions proches des gènes. Au contraire, les contacts de chromatine peuvent permettre d’associer des régions à très grandes distances, mais il est délicat de considérer avec une grande fiabilité les interactions à courtes distances. En effet, deux loci proches linéairement sur le génome ont une forte probabilité d’être proche spatialement dans le noyau. On observe ainsi que la distance génomique entre deux loci est corrélée positivement à la distance physique dans le noyau. Les traitements statistiques des contacts de \acrshort{PCHi-C} prennent en compte ce biais pour définir les interactions significatives, mais la proportion de faux positifs reste élevée pour les contacts à courte distance \citep{cairns_chicago_2016}. En éliminant par précaution les contacts à courte distance, la distribution des distances entre gènes et éléments \gls{cis}-régulateurs est différente selon une approche par \acrshort{PCHi-C} et par voisinage. Ces associations sont cependant biologiquement pertinentes au vu des fortes corrélations entre l’expression des gènes et les activités des \glspl{amplificateur} à courte distance \citep{laverre_long-range_2022}. Le modèle de régulation à courte distance très largement utilisé reste donc pertinent dans de nombreux cas. Néanmoins les contacts à grande distance révèlent également des fonctions cohérentes, confirmant certaines associations par voisinage mais en révélant de nouvelles. Ainsi dans GOntact nous proposons une méthode "hybride" d'association entre gène et éléments \gls{cis}-régulateurs pour tirer partie de la présence de relations régulatrices dans le voisinage et à grande distance des promoteurs des gènes. \\

\subsubsection*{Nombreuses méthodes d'inférence de relations régulatrices}

De nombreuses méthodes d’asociation entre gène et éléments \gls{cis}-régulateurs ont été développées que ce soit par voisinage (à la manière de GREAT \citet{mclean_great_2010}), par conformation de la chromatine (\acrshort{PCHi-C}), par conservation évolutive de la synténie entre gènes et éléments \gls{cis}-régulateurs (PEGASUS: \citet{clement_enhancergene_2020}), par co-activité (FOCS : \citet{hait_focs_2018}), par corrélation de l’accessibilité de la chromatine (Cicero: \citet{pliner_cicero_2018}, SnapATAC: \citet{fang_comprehensive_2021}) ou encore par la présence de marques épigénétiques identiques (TargetFinder : \citet{whalen_enhancerpromoter_2016}), pour ne citer que quelques exemples. Chacune permet d’obtenir un aperçu différent de l’organisation des paysages \gls{cis}-régulateurs des gènes tout en comportant des biais techniques ou inhérents aux méthodes. Multiplier les approches et construire une base de données qui intégrerait les prédictions d’éléments \gls{cis}-régulateurs et leurs potentiels gènes cibles au fur et à mesure des publications pourrait permettre de caractériser de manière systématique les paysages \gls{cis}-régulateurs des gènes. De tels outils commencent à émerger par exemple avec GeneHancer \citep{fishilevich_genehancer_2017}. Celui-ci utilise les prédictions d’\glspl{amplificateur} de l’humain provenant de quatre sources différentes ainsi que trois approches distinctes pour l’association entre gènes et amplificateurs. Il intègre notamment les résultats de la première étude de \acrshort{PCHi-C} \citep{mifsud_mapping_2015}. GeneHancer calcule ainsi un score de confiance pour chaque \gls{amplificateur} et pour chaque association avec un gène selon le nombre de sources différentes qui l'identifient. Cette méthode permet d’obtenir un large nombre d’associations mais nécessiterait d’intégrer de nombreuses données supplémentaires pour comprendre les mécanismes de la régulation de l’expression des gènes dans un grand nombre de contextes. L’accumulation de données nécessite de collaborer à l’échelle internationale et de proposer des initiatives communes pour démêler et rendre cohérents les efforts dans la compréhension des génomes.

\section{Corrélation entre l’évolution de l’expression des gènes et des paysages \textit{cis}-régulateurs}

Le second objectif de ma thèse a été d'analyser l'évolution des paysages \gls{cis}-régulateurs chez les vertébrés à partir des contacts de la chromatine  et d'analyser ses relations avec l'évolution de l'expression des gènes à partir de comparaisons entre l'humain et la souris. 

\subsection{Contacts de chromatine conservés entre humain et souris}
Mes travaux ont montré que les contacts de la chromatine centrés sur les promoteurs des gènes sont plus conservés qu’un attendu neutre entre des espèces distantes comme l’humain et la souris. Les régions contactées sont également conservées en séquence et en synténie avec leurs gènes cibles à l'échelle des vertébrés. Des structures homologues séparées par de grandes distances génomiques sont significativement maintenues en contact dans des types cellulaires qui ne sont pas nécessairement les mêmes. De plus, les gènes dont les paysages \gls{cis}-régulateurs sont les plus conservés sont enrichis en gènes dont le patron d’expression est fortement contraint comme les gènes du développement. Cette observation donne un argument supplémentaire à la pertinence fonctionnelle des interactions de la chromatine mesurées à cette échelle. La conservation des contacts de chromatine est également en accord avec ce qui a déjà été observé pour l’évolution des domaines d’associations topologiques (\acrshort{TAD}s), des régions génomiques où les contacts sont favorisés \citep{dixon_topological_2012, harmston_topologically_2017, krefting_evolutionary_2018}. \\

Ceci indique également que d’importantes pressions de sélection contraignent l’évolution à grande échelle de l’organisation génomique. Les réarrangements perturbant ces relations régulatrices sont contre-sélectionnés comme cela a déjà été montré par les analyses de la distribution des points de cassure dans le génome \citep{lemaitre_analysis_2009, swenson_large-scale_2019}. L’importante conservation de la synténie entre les gènes et les éléments \gls{cis}-régulateurs à grande distance corréle effectivement avec l’organisation spatiale du génome \citep{clement_enhancergene_2020}. La prise en compte de l'organisation tridimensionnelle du génome et des paysages \gls{cis}-régulateurs est ainsi importante pour comprendre l’impact des réarrangements sur l'expression des gènes. Les réarrangements entre un gène et un élément \gls{cis}-régulateur n’entraînent en effet pas nécessairement une perturbation du contact \citep{symmons_SHH_2016}. Des promoteurs et amplificateurs séparés pas des distances linéaires très différentes entre deux espèces peuvent-être spatialement proches et ainsi conserver une relation régulatrice homologue \citep{veron_close_2011}. Le maintien des loci dans une certaine gamme de distance est possible, ce qui explique que l’évolution de la distance à l’intérieur des \acrshort{TAD}s suivrait une dynamique évolutive similaire à l’évolution de la taille des introns \citep{clement_enhancergene_2020}. Analyser conjointement les contacts de chromatine et les réarrangements pourraient permettre d’expliquer les mécanismes moléculaires à l’origine de l’évolution de l’expression de certains gènes. 

\subsection{Absence de corrélation avec l’évolution du niveau d’expression des gènes}

Nous n’avons pas observé de corrélation entre l’évolution des paysages \gls{cis}-régulateurs, en terme de séquence, de synténie ou de contact de chromatine, et l’évolution du niveau moyen d’expression dans des types cellulaires comparables entre l’humain et la souris. Ces différences quantitatives sont difficiles à mesurer entre des espèces aussi distantes que l’humain et la souris, d’autant plus que nous avons analysé des données provenant de différentes sources. Les niveaux d’expression (RNA-seq), la prédiction d’éléments \gls{cis}-régulateurs (combinés sur de nombreux échantillons) et les informations de contact (\acrshort{PCHi-C}), proviennent d’études différentes sur des tissus différents et donc comprennent une source importante de biais. \\

Cette observation pourrait également être en accord avec les précédentes études qui ont montré des paysages \gls{cis}-régulateurs évoluant rapidement (par leur séquence et leur activité) \citep{cheng_principles_2014, villar_enhancer_2015}, contrairement aux niveaux d’expression qui évoluent lentement chez les vertébrés \citep{brawand_evolution_2011, cardoso-moreira_gene_2019, berthelot_complexity_2018}. La redondance fonctionnelle des éléments régulateurs pourrait être un mécanisme important pour expliquer cette robustesse d’expression \citep{berthelot_complexity_2018, osterwalder_enhancer_2018, kvon_enhancer_2021}. Au niveau de l’évolution de la séquence, la redondance fonctionnelle permettrait de relâcher les pressions de sélection sur chacun des éléments \gls{cis}-régulateurs. Cette forme de compensation des mutations serait d’autant plus importante que le paysage \gls{cis}-régulateur est complexe. Cette compensation pourrait également s’effectuer à différentes échelles par les nombreux autres mécanismes régulateurs de l’expression. Par exemple, il a été montré que de nombreuses mutations en cis pouvaient être compensées par des variations de la régulation en \gls{trans} \citep{goncalves_extensive_2012, mack_gene_2016}. De plus, la faible spécificité des motifs reconnus par les facteurs de transcription serait en mesure d’assurer la conservation de la fonction régulatrice d’une séquence divergente. Certains éléments \gls{cis}-régulateurs fortement conservés en micro-synténie avec leurs gène cibles à l’échelle des métazoaires semblent par exemple avoir conservé leur fonction tout en ayant suffisamment divergé pour n’être plus alignables entre espèces \citep{wong_deep_2020}. Cette importante dégénérescence du “code de régulation” des éléments \gls{cis}-régulateurs doit être investiguée pour mieux comprendre l’évolution des mécanismes régulateurs de l’expression des gènes. A la manière des séquences codantes, celà permettrait d’identifier des mutations “silencieuses” ou non pour la fonction des éléments \gls{cis}-régulateurs et ainsi potentiellement déceler des changements de forces sélectives. Cela pourrait également permettre d’expliquer l’importante variabilité des conséquences de l’évolution des séquences \gls{cis}-régulatrices. En effet, des cas de perturbation extrême comme une délétion complète d’éléments \gls{cis}-régulateurs fonctionnels sans conséquence phénotypique \citep{osterwalder_enhancer_2018} co-existent avec des cas où de simples mutations ponctuelles engendrent un important changement morphologique ou sont associées à des maladies \citep{corradin_enhancer_2014, kvon_progressive_2016}. 

\subsection{Corrélation avec l’évolution du patron d’expression des gènes}
La différence d’échelle entre la prédiction des paysages \gls{cis}-régulateurs modulables obtenus à partir des données de \acrshort{PCHi-C} et les niveaux d’expression mesuré dans un seul échantillon révèle une nécessité d’observer un tableau plus complet de l’activité des gènes. C’est la raison pour laquelle nous avons analysé les patrons de l’expression des gènes à travers plusieurs organes et stades de développement comparables entre l’humain et la souris \citep{cardoso-moreira_gene_2019}. Les différences mesurées sont donc davantage qualitatives et sont plus à même de révéler des changements majeurs comme des gains ou des pertes d’expression dans une \gls{condition}. \\

Nous avons observé une corrélation entre l’évolution des paysages \gls{cis}-régulateurs et l’évolution du patron d’expression des gènes. Les \glspl{amplificateur} les plus conservés en séquence, mais également ceux maintenus en synténie avec les gènes ou encore en contact de chromatine conservé entre l’humain et la souris sont associés à des gènes aux patrons d’expression conservés. Cette observation est également en accord avec ce qui avait été observé pour les gènes à l’intérieur des \acrshort{TAD}s conservés à l’échelle des vertébrés qui sont associés à une plus grande conservation de leur patron d’expression entre humain et souris \citep{krefting_evolutionary_2018}. Analyser l’expression des gènes dans un grand nombre de contextes permettrait d’augmenter la probabilité d’observer les variations du paysage \gls{cis}-régulateur qui impactent l’expression des gènes. En effet, de par la compensation des mécanismes mais également de la pré-formation de la structure de la chromatine, le gain ou la perte d’un contact entre un gène et un éléments \gls{cis}-régulateur n’entraine pas nécessairement une modification de son l’expression dans l’ensemble des contextes. 

\subsection{Analyse d’autres catégories de gènes}
Dans ce travail, nous avons seulement considéré les gènes codants pour des protéines. Il serait également intéressant d’inférer les paysages \gls{cis}-régulateurs des gènes non-codants, notamment les loci produisant de longs ARNs non-codants, dont la régulation et les rôles sont encore largement incompris \citep{mercer_long_2009, ransohoff_functions_2018}. Ces gènes peuvent notamment être impliqués dans de nombreux processus de régulation de l’expression d’autres gènes à différentes échelles, pré et post-transcriptionnelles. Ils peuvent agir sur l’expression des gènes en \gls{cis}, par l’action directe dans le voisinage proche ou potentiellement par contact de chromatine à plus longue distance. Les données de \acrshort{PCHi-C} que j’ai analysées devraient permettre d’étudier l'évolution de leurs paysages \gls{cis}-régulateurs entre l’homme et la souris, et peut-être même de prédire leurs gènes cibles, si la régulation se fait par contact de chromatine.\\

De plus, nous avons restreint nos analyses aux gènes qui sont orthologues un-à-un entre les espèces. Nous avons donc exclu deux classes de gènes intéressants : ceux qui ont subi des duplications depuis la divergence des deux espèces, et ceux qui ont été perdus (pseudogénisés) au cours de l’évolution dans l’une des deux lignées. Ces deux catégories de gènes sont intéressantes pour des raisons différentes. D’un côté, l’évolution de l’expression des gènes après duplication a attiré beaucoup d’attention \citep{brunet_gene_2006, guschanski_evolution_2017, carelli_repurposing_2018}. De nombreux cas de changements de patron d’expression entre les copies des gènes après duplication ont été mis en évidence. Les changements des paysages \gls{cis}-régulateurs qui sont associés à ces changements d’expression ne sont pas encore parfaitement compris, et les données de \acrshort{PCHi-C} pourraient aider pour répondre à cette question. De l’autre côté, en ignorant les cas de pseudogénisation nous excluons une potentielle conséquence majeure des changements des paysages \gls{cis}-régulateurs : des cas extrêmes de changement de régulation (tels que ceux entraînés par de grands réarrangements génomiques) pourraient mener à une perte de fonction des gènes et donc à une pseudogénisation. Inversement, une forte divergence de certains éléments \gls{cis}-régulateurs pourrait être expliquée par la perte de fonction des gènes cibles. La prise en compte de ces cas de pseudogénisation permettrait de mieux saisir les implications phénotypiques de l’évolution des mécanismes \gls{cis}-régulateurs.

\section{Relation avec l’évolution phénotypique}

Finalement, mon dernier objectif a été d'investiguer les relations entre l'évolution des paysages \gls{cis}-régulateurs, l'évolution de l'expression des gènes et l'évolution phénotypique. En étudiant la perte convergente du phallus chez plusieurs lignées d’oiseaux, nous souhaitions mieux comprendre les déterminants génétiques de ce changement phénotypique majeur.

\subsection{Patrons d’expression conservés malgré les changements phénotypiques}

Premièrement, en comparant des profils d’expression au cours du développement du tubercule génital nous avons montré que l’expression de la grande majorité des gènes suit une trajectoire temporelle similaire entre le poulet et le canard. Cette observation est en accord avec l’évolution lente de l’expression des gènes observés dans plusieurs études malgré d’importantes variations morphologiques \citep{brawand_evolution_2011, cardoso-moreira_gene_2019}. Le caractère fortement pléiotropique des gènes exprimés au cours du développement est généralement mis en avant pour expliquer ces résultats. Dans le cas du développement du tubercule génital, il est en effet connu que de nombreuses voies de signalisation sont partagées avec le développement des membres chez la souris \citep{lonfat_convergent_2014,infante_shared_2015}. De plus, comme nous l’avons vu tout au long de cette thèse avec l’exemple de \acrshort{SHH}, le changement d’expression de peu de gènes peut avoir un impact majeur sur les phénotypes. Nous avons ainsi détecté des gènes qui ont une dynamique temporelle différente entre les deux espèces. Par exemple, on observe une diminution du niveau d'expression de \textit{HOXD13} au cours du développement chez le poulet alors que son expression est en forte croissance avec le temps chez le canard.  Ce gène est connu pour jouer un rôle important dans le développement des organes génitaux chez l'humain et la souris \citep{dolle_hox-4_1991, klonisch_molecular_2004}. Dans ces gènes candidats, nous ne retrouvons pas le gène \textit{BMP4}, qui avait été proposé sur la base d'expérience d’hybridation \textit{in situ} et de variations quantitatives \citep{herrera_developmental_2013}. \\

Pour confirmer nos résultats, il est essentiel de valider et de quantifier plus précisément les patrons d’expression des gènes candidats. Patrick Tshopp et  Maëva Luxey, nos collaborateurs, comptent pour cela produire des données de transcriptomique en cellule unique pour déterminer dans quel type cellulaire du tubercule génital ces gènes sont exprimés. Ces données permettraient de comprendre avec plus de détails les mécanismes cytologiques qui affectent le développement de cet organe. Il serait par exemple intéressant d’identifier à partir de quelle région du tubercule génital l’apoptose est déclenchée ou si une voie de signalisation se localise particulièrement dans certaines cellules. Pour répondre à cette question, des expériences d’hybridation \textit{in situ} sont également envisageables pour visualiser plus largement les patrons d’expression de gènes candidats. 

\subsection{Détection des séquences \textit{cis}-régulatrices}

Afin de prédire les éléments \gls{cis}-régulateurs de l’expression des gènes dans les génomes d’oiseaux nous avons utilisé des données d’ouverture de la chromatine (ATAC-seq) de plusieurs échantillons du poulet et du canard. Nous avons également identifié des séquences non-codantes qui sont particulièrement conservées au sein des oiseaux possédant un phallus ; certaines de ces séquences pourraient avoir des rôles régulateurs \citep{sackton_convergent_2019, zhu_three_2021}. Grâce à des analyses comparatives des génomes alignés de plusieurs espèces d’oiseaux \citep{feng_dense_2020} nous avons pu mettre en évidence des séquences non-codantes potentiellement impliquées dans le développement du phallus chez les oiseaux. En effet, nous avons pu identifier des séquences dont le taux d’évolution est accéléré de façon convergente dans les espèces ayant perdu le phallus. Ce patron d’évolution est attendu pour des séquences qui seraient impliquées dans la mise en place du phallus et qui auraient subi une relaxation de pressions de sélection purifiante suite à la perte de cet organe \citep{hiller_forward_2012}. De plus amples analyses des motifs de fixation des facteurs de transcription affectés par la divergence des séquences accélérées pourraient permettre de proposer des fonctions régulatrices possibles de ces régions. \\

Malgré la qualité hétérogène des données d’ATAC-seq utilisées, nous avons pu détecter un enrichissement de ces séquences accélérées actives dans le tubercule génital du canard. De futures données d’ATAC-seq pourraient permettre de confirmer cette observation en identifiant plus rigoureusement les séquences actives spécifiquement dans le tubercule génital. De plus, elles permettraient de comparer plus précisément l’activité des séquences dans le tubercule génital du poulet et du canard. En effet, il serait envisageable que la divergence de séquence ne soit pas le seul mécanisme impliqué dans les changements d’expression des gènes responsable de ce phénotype. Les variations d’activité d’éléments \gls{cis}-régulateurs conservés pourraient également participer à la mise en place de patron d’expression divergent entre le poulet et le canard \citep{villar_enhancer_2015}. \\

Des analyses conjointes entre les gènes différentiellement exprimés chez le poulet et le canard et les séquences accélérées sont également nécessaires pour tenter de comprendre les voies métaboliques affectées par ces variations. Nous avons pu observer que les gènes associés à des séquences accélérées dans les génomes d’oiseaux ayant perdu le phallus sont enrichis en fonction du développement de l’organe génital mâle. Ces séquences sont donc particulièrement intéressantes et pourraient révéler des interactions \gls{cis}-régulatrices importantes pour la mise en place du phallus. Cependant, en raison de contraintes de temps, nous avons uniquement prédit l’association entre les séquences identifiées et leurs gènes cibles par une approche de voisinage. Comme nous l’avons vu tout au long de cette thèse, l’association entre gène et éléments \gls{cis}-régulateurs par cette approche pourrait manquer des relations régulatrices importantes. Des données de capture de conformation de la chromatine sont disponibles chez le poulet et le canard. Leur prise en compte semble particulièrement importante dans le contexte où les gènes du développement présentent de nombreuses interactions à grandes distances difficilement prédictibles par voisinage \citep{montavon_landscapes_2012, de_laat_topology_2013}. Des techniques de conformation de la chromatine ciblées sur certains gènes (comme le 4C par exemple) pourraient aussi permettre d’évaluer avec une plus grande précision les paysages \gls{cis}-régulateurs de gènes candidats \citep{simonis_nuclear_2006,gondor_high-resolution_2008}.\\

Par la suite, des validations expérimentales sont nécessaires pour valider le potentiel régulateur de ces séquences ainsi que leur patron d’activité dans le tubercule génital. Avec une liste de séquences candidates réduites il serait ainsi envisageable d’effectuer des expériences d’hybridation \textit{in situ} ou encore des modifications génomiques (délétion ou transgénèse) sur des embryons de canard et de poulet pour observer leur impact sur le développement du phallus.

\subsection{Évolution du répertoire des gènes et des séquence codantes}

Les avancées bio-moléculaires permettent de mieux comprendre le fonctionnement des génomes et révèlent la multiplicité et la complexité des bases moléculaires à l’origine des changements phénotypiques. Nous n’avons ici analysé qu’une partie des mécanismes possibles pour expliquer l’évolution du phallus chez les oiseaux. En raison d’un attendu théorique fort sur l’importance de la contribution des variations des séquences \gls{cis}-régulatrices sur les changements morphologiques, nous avons mis de côté l’analyse des séquences codantes. Pour autant, de nombreux exemples d’évolution morphologique ayant pour origine probable des variations de ces séquences sont décrits \citep{stern_loci_2008, burga_genetic_2017, sharma_genomics_2018}. \\

Il serait ainsi important d'analyser d'une part les variations du répertoire de gènes (protéiques ou ARNs non-codants) des génomes. Les gains, les duplications ou les pertes de gènes sont en effet des changements majeurs qui peuvent impacter le fonctionnement des organismes et qui sont des contributeurs importants à l’évolution phénotypique des vertébrés \citep{kaessmann_origins_2010, chen_new_2013, albalat_evolution_2016}. La duplication de gène peut par exemple entraîner une augmentation de la quantité de protéines produites ou favoriser l’apparition de nouvelles fonctions suite à l’accumulation des mutations dans les deux copies du gène. La contribution des pertes de gènes pourrait également être importante dans les cas de pertes phénotypiques. De telles pertes seraient notamment impliquées dans plusieurs convergences évolutives chez les mammifères comme la production d’écailles chez le pangolin et le tatou, l’adaptation à la vie aquatique chez les cétacées et les siréniens, ou la perte de l’émail des dents dans plusieurs lignées \citep{sharma_genomics_2018, huelsmann_genes_2019}. D'autre part, il serait intéressant d'estimer le taux d’évolution des séquences codantes au sein de chaque lignée pour détecter des potentiels changements de pressions de sélection agissant sur les gènes \citep{nei_simple_1986}. En effet les mutations affectant la séquence codante sont en mesure de modifier le produit des gènes et pourrait ainsi participer à l’établissement de changements phénotypiques convergents. Les gènes présentant des variations de taux d’évolution similaires chez des espèces au phénotype convergent pourraient alors être de bons candidats pour expliquer l’évolution de ce phénotype. Finalement, la détection de changements plus fins comme une sélection positive sur un faible nombre de sites de la protéine peut également être évaluée à partir des comparaisons de profils d’acides aminés sur les séquences protéiques \citep{rey_detecting_2019}. De telles substitutions convergentes seraient par exemple impliquées dans la convergence de l’écholocalisation chez des chauve-souris, le dauphin et l’orque \citep{marcovitz_functional_2019}. De nombreux outils, notamment développées par nos collégues au LBBE, sont à notre disposition et seraient pertinents pour analyser ce type de convergence sur les séquences protéiques. \\
