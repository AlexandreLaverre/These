\chapter*{Remerciements}
\label{remerciements}

Je souhaite remercier tout d’abord Camille Berthelot, Mélanie Debiais-Thibaud, Hugues Roest-Crollius et Cristina Vieira d’avoir accepté d’évaluer mes travaux de thèse.\\

Je voudrais également remercier les personnes qui ont participé à mes trois comités de suivi de thèse Yad Ghavi-Helm, Franck Picard, Dominique Mouchiroud, ma tutrice Clémentine François et encore une fois Camille Berthelot pour votre écoute, vos conseils avisés et votre bienveillance.\\

Je n'aurais jamais pu faire cette thèse sans mes superbes encadrant.e.s Anouk et Eric. Merci à vous deux de m'avoir proposé ce stage et de m'avoir suivi avec bienveillance ces quatres dernières années. Anouk, merci infiniment d'avoir cru en moi depuis le tout début et d'avoir eu cette envie de m'accompagner pour une thèse. J'ai tellement appris avec toi, je n'aurais jamais pu rêver d'un meilleur encadrement. Merci pour ta disponibilité, ta patience et ta gentillesse. Comme dirait Théo, tu es maintenant ma maman de la Science. \\

Je tiens ensuite à remercier tous les membres de mon équipe de bioinformatique, phylogénie et génomique évolutive pour leur sagesse et leur contribution majeure à mon ouverture scientifique. Je remercie particulièrement Laurent Duret pour ses conseils toujours pertinents et pour m’avoir permis de franchir les portes du LBBE en faisant circuler mon CV dans le labo quand je cherchais un stage.\\

Je souhaite remercier l’ensemble des personnes avec qui j’ai eu l’occasion de travailler sur différents projets au cours de ma thèse. Merci à Nelly Burlet pour sa disponibilité, ses conseils et pour nous avoir permis d’enfiler une blouse blanche, petit.e.s bio-informaticien.ne.s que nous sommes, dans nos tentatives d’extraction d’ADN de plumes. Merci à Marie Fablet et Timothée Kastylevsky pour leur contribution à l’analyse des éléments répétés des génomes d’oiseaux. Merci à Etienne Rajon, Florian Labourel, Mariana Ferrarini et Sergio Peignier pour cette initiative de recherche sur ACE2 au moment où le monde entier basculait dans un premier confinement. Vous m’avez permis de retrouver un peu de sens dans mon activité de recherche à un moment crucial. Merci beaucoup à Patrick Tschopp et Maëva Luxey de l'université de Bâle pour leur implication sans faille dans l'étude des phallus d'oiseaux et les nombreux échanges toujours très constructifs. \\

Merci également aux personnes qui m'ont permis de décrouvrir le monde de l'enseignement, Véronique Daviero, Frédéric Thévenard, Anne-Kristel Bittebière, Arnaud Mary, et surtout merci à l'ensemble de mes étudiant.e.s qui m'ont donné le goût de transmettre des connaissances. Les TPs à balader dans les serres botaniques pour raconter la vie des plantes vont bien plus me manquer que les TPs de biostats en visio ! \\

Merci à toutes les personnes du pôle informatique du LBBE, particulièrement Bruno Spataro et Stéphane Delmotte que j'ai souvent embêté pour des soucis de cluster, de pbil ou de stockage ! Merci pour le support sans faille et les nombreux conseils que vous apportez à tout ce labo. Merci également à Christophe Blanchet de l'Institut Français de Bio-informatique pour sa gentillesse et pour m'avoir débloqué tous mes problèmes de VM. Merci aussi aux personnes du pôle administratif qui font tourner le labo et nous sauvent dans toutes nos démarches. \\

Merci aux différentes personnes qui ont partagé mon bureau durant ces 4 années, Christine Oger, Philippe Weber, Claire Gayral, Adil El Filali qui ont su animer mon quotidien. Un énorme merci à Alexia, sans qui la vie au labo serait moins fun ! Tu as toujours ce petit grain de folie qui redonne le sourire aux gens. Avec toi on a presque réussi à faire un petit jardin dans ce bureau !\\

Je voudrais aussi remercier l'ensemble des doctorant.e.s, stagiaires, post-doc du LBBE qui ont croisés ma route. Vous êtes la force de ce labo, on se demande comment la science avancerait sans vous, ne l'oubliez pas ! Théo, Djivan, Alexia, on a formé une vraie petite équipe depuis notre arrivée dans ce couloir pour un stage de M2, merci pour tous ces moments. (En fait non, je tiens surtout à ne PAS remercier Théo ;)) Un très grand merci à Thibault Latrille qui a été notre doctorant de référence à tous, qui a toujours trouvé du temps pour répondre à toute nos questions. Je le remercie tout particulièrement pour son template LateX qui m'a occupé la plupart de mes heures ces dernières semaines. Merci à celles et ceux qui sont déjà partis ou qui ont fait des passages trop rapides au LBBE : Pierre, Elise, Maud, Monique, Lucie, Claire, Marina, Diego et j'en oublie sans doute, ne m'en voulez pas! Merci à Mary, Kamal et \'Emilie pour ces instants partagés même dans les plus grand moments de doute dans nos thèses respectives. Merci à toutes les personnes qui mettent de l'ambiance dans ce labo, qui sont toujours partantes pour boire des coups ou faire des pauses café, je pourrai pas citer tout le monde mais merci ! Je souhaite un bon courage aussi à celles et ceux qui sont sur la fin et à la relève : Aissa, Florentin, Chloé, Benjamin, Julien, Laura, Léa, Solène, Lucas, Mélodie et bien d'autres. Pour terminer avec les "non-permanent.e.s", je voudrais vraiment remercier les personnes qui ont partagé l'aventure des Pinsons MigRateurs, au sein du bureau de l'asso, mais aussi de toutes les animations qu'on a pu initier ou pérenniser : week-ends d'inté, séminaires post-thèse, séminaires étudiants, mais aussi les regrettées HappyHours avec nos LBBbières ratées ! Vous avez toute ma confiance pour continuer à faire vivre la vie étudiante (et scientifique) du labo !\\

Merci à tous les gens qui ont partagé mon aventure lyonnaise en dehors du labo. Merci à mes cher.e.s colocs Léa, Eve et Camille avec qui j'ai partagé près de 2 ans de ma vie mais aussi le premier confinement (ça compte triple!). A toutes les belles rencontres que j'ai pu faire : Anaïs, Judith, Camille, Elsa, Romain, Mégane, Coralie, Julia, Clémentine mais aussi toute la Team des Graines électroniques !\\

Merci à tous mes ami.e.s de Montpellier pour le soutien moral et émotionnel, on ne se voit pas assez mais je n'aurais jamais pu tenir sans vous. Un énorme merci à Aurélien et Maurine que j'aime de tout mon coeur. Merci à toute la coloc' de Celleneuve: Maurine, Nathan, Arthur, Taïna, Salma, Léo-Paul, à la coloc' de choc : Aurélien et Thibault, ainsi qu'à Romain pour vraiment tout ce que vous apportez dans ma vie et aussi pour m'avoir toujours accueilli avec plaisir pour une nuit, un week-end (parfois transformée en semaine!) sur Montpellier. J'ai aussi une grosse pensée pour la petite bande du lycée maintenant bien dispersée : Fred, Charlotte, Debora, Gaetan et Julie, vous me manquez ! Merci à tous les copains et les copines que je n'ai pas cités, vous comptez beaucoup !\\

Je voudrais évidemment remercier ma petite famille qui me soutiendra toujours dans tout ce que je fait et dans toutes mes décisions. Maman, Papa, Nath, Nanou, Romain et les petits loups!\\

Finalement, le plus important, merci à mon amoureux, Jordan, qui me rend toujours plus heureux, qui me soutien sans faille depuis 4 ans et qui me donne la force et le courage au quotidien. Merci à toi d'avoir supporté cette thèse autant que moi ! Je t'aime. 



