\chapter{Objectifs de cette thèse}
\label{chap:objectifs}

Au cours de cette thèse, je me suis d’abord intéressé à définir les paysages \gls{cis}-régulateurs de l’expression des gènes chez l’humain et la souris à partir des contacts de la chromatine et de prédictions d’éléments \gls{cis}-régulateurs. Dans le Chapitre 2, mon objectif a été de comparer ces paysages \gls{cis}-régulateurs avec ceux obtenus par les méthodes les plus généralement employées qui consistent à associer les gènes aux éléments \gls{cis}-régulateurs voisins. Je me suis ensuite attaché à savoir si les contacts de chromatine pouvaient permettre d’attribuer des fonctions différentes aux éléments \gls{cis}-régulateurs.\\

Dans le Chapitre 3, je me suis demandé comment les paysages \gls{cis}-régulateurs tels que mesurés par les contacts de chromatine évoluent chez plusieurs espèces de vertébrés. Particulièrement, j’ai cherché à étudier si les paysages \gls{cis}-régulateurs imposent des contraintes sur l’évolution de l’organisation des génomes. J’ai également analysé les associations entre la structure des paysages \gls{cis}-régulateurs et l’expression des gènes en termes de niveaux et d’étendue de l’expression au sein de plusieurs tissus et stades embryonnaires chez l’humain et la souris. J’ai ensuite investigué les relations entre l’évolution des profils d’expression des gènes et l’évolution des paysages \gls{cis}-régulateurs mesurés par la divergence des séquences, de synténie entre gène et éléments \gls{cis}-régulateur et des contacts de chromatine.\\

Finalement dans le Chapitre 4, j’ai cherché à comprendre comment l’évolution du paysage \gls{cis}-régulateur peut impacter l’évolution d’un trait phénotypique. Je me suis pour cela attaché à comprendre la relation entre l’évolution des éléments \gls{cis}-régulateurs et la perte convergente du phallus chez les oiseaux. Grâce aux profils d’expression des gènes chez le poulet et le canard ainsi qu’à une étude comparative des génomes de plusieurs oiseaux, j’ai cherché à détecter des séquences \gls{cis}-régulatrices qui pourraient expliquer ce changement phénotypique majeurs survenant lors du développement. Pour cette étude, le génome d’un oiseau nouvellement séquencé a été utilisé, j’ai donc assemblé son génome et analysé ses caractéristiques.