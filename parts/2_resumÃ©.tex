\chapter*{Résumé}
\label{résumé}

Chez les organismes multicellulaires, la régulation de l'activité des gènes est nécessaire pour établir des profils d’expression spécifiques garantissant le développement et la diversité fonctionnelle de l’organisme. Des variations de la régulation des gènes pourraient expliquer une partie importante des différences observées entre les espèces, en permettant d’affiner l’activité des gènes sans modifier leur séquence. L’expression des gènes est régulée par des interactions moléculaires complexes, impliquant notamment des éléments régulateurs agissant en \textit{cis}, c’est-à-dire se trouvant sur le même chromosome que leurs gènes cibles. L’identification de ces éléments dans les génomes est désormais possible grâce au développement de nombreuses techniques biomoléculaires. Cependant, définir les associations entre les gènes et les éléments \textit{cis}-régulateurs reste un défi, d’autant plus que ces derniers peuvent se situer à grande distance génomique des gènes qu’ils contrôlent. Des boucles de chromatine mettent en interaction physique les éléments \textit{cis}-régulateurs distaux et les promoteurs des gènes. Il est aujourd’hui possible de détecter ces boucles avec des méthodes de capture de la conformation de la chromatine. De données s’accumulent et permettent de mieux définir les paysages \textit{cis}-régulateurs et de comprendre les implications des contacts de chromatine dans l’expression des gènes et son évolution.\\

Durant cette thèse, j’ai d’abord combiné et analysé des cartes de contacts de la chromatine obtenues pour plusieurs types cellulaires chez la souris et chez l’humain. J’ai comparé les paysages \textit{cis}-régulateurs définis par les approches classiques de voisinage et par les contacts de la chromatine en développant GOntact. Cet outil permet d’inférer un enrichissement fonctionnel pour un ensemble d’éléments régulateurs, à partir de l'annotation fonctionnelle des gènes contactés par ces éléments. En utilisant ces données de contacts, j’ai ensuite montré qu'il existe une conservation évolutive significative des paysages \textit{cis}-régulateurs chez les vertébrés, indiquant que la sélection naturelle agit pour préserver non seulement les séquences des éléments régulateurs mais aussi leurs contacts chromatiniens avec les gènes cibles. Nous avons montré que l'évolution des paysages \textit{cis}-régulateurs, mesurée en termes de séquences des éléments, de synténie ou de contacts avec des gènes cibles, est significativement associée à l'évolution de l'expression des gènes. \\

Finalement, j’ai cherché à évaluer la relation entre l’évolution des éléments \textit{cis}-régulateurs et l’évolution phénotypique des organismes. J’ai étudié la perte convergente du phallus intervenue chez plusieurs lignées d’oiseaux. Le déterminisme génétique de ce changement phénotypique est encore inconnu mais semble impliquer des modifications importantes des patrons d’expression de gènes du développement, dont la séquence est très conservée. J'ai réalisé une étude comparative sur les génomes complets d'oiseaux, en utilisant des données moléculaires mesurant l'expression des gènes et détectant l’ouverture de la chromatine. En particulier, j’ai analysé l’évolution des séquences potentiellement régulatrices. Ceci nous a permis de proposer des régions candidates associées à ce changement phénotypique. \\

Les résultats obtenus au cours de cette thèse ont permis de confirmer l’importance de l’utilisation des contacts de la chromatine pour associer les promoteurs des gènes à leurs éléments régulateurs dans un contexte évolutif. En redéfinissant les paysages \textit{cis}-régulateurs, nous avons montré que leur évolution permet d’expliquer l’évolution de l’expression des gènes et que des nouvelles fonctions peuvent être attribuées aux éléments régulateurs. Nos résultats confirment également que l'étude de l'évolution de ces éléments \textit{cis}-régulateurs peut fournir des explications à des changements morphologiques majeurs.\\

\chapter*{Abstract}
\label{abstract}
