\newacronym{3C}{3C}{Capture de la Conformation des Chromosomes}
\newacronym{ADN}{ADN}{Acide Désoxyribo-Nucléique}
\newacronym{ARN}{ARN}{Acide Ribo-Nucléique}
\newacronym{ARNm}{ARNm}{Acide Ribo-Nucléique messager}
\newacronym{ATACseq}{ATACseq}{Assay for Transposase-Accessible Chromatin sequencing}
\newacronym{CAGE}{CAGE}{Cap Analysis of Gene Expression}
\newacronym{ChIP-seq}{ChIP-seq}{Chromatin Immuno-Precipitation sequencing ou séquençage par Immunoprécipitation de la Chromatine}
\newacronym{CpG}{CpG}{Cytosine–phosphate-Guanine ou dinucléotide Cytosine-Guanine}
\newacronym{CTCF}{CTCF}{Facteur de transcription se fixant au motif CCCTC}
\newacronym{eRNA}{eRNA}{enhancer Ribo-Nucléique Acid ou Acide Ribo-Nucléique produit par les éléments amplificateurs de l'expression}
\newacronym{FT}{FT}{Facteur de Transcription}
\newacronym{GFP}{GFP}{Green Fluorescent Protein ou protéine fluorescente verte}
\newacronym{Hi-C}{Hi-C}{High-throughput technique to capture chromatin conformation}
\newacronym{PCHi-C}{PCHi-C}{Promoter Capture Hi-C}
\newacronym{PCR}{PCR}{Polymerase Chain Reaction}
\newacronym{RNA-seq}{RNA-seq}{Ribo-Nucléique Acid sequencing ou séquençage des Acides Ribo-Nucléiques}
\newacronym{RPKM}{RPKM}{Read Per Kilobase per Million ou lecture de séquençage par kilobase par million}
\newacronym{SHH}{SHH}{Sonic HedgeHog}
\newacronym{TAD}{TAD}{Topological Associated Domain ou domaine d'association topologique}
\newacronym{TSS}{TSS}{site d'initiation de la transcription}
\newacronym{TG}{TG}{Tubercule Génital}
\newacronym{ZRS}{ZRS}{Zone of polarising activity Regulatory Sequence ou séquence régulatrice de l'activité de la zone polarisante}


\newglossaryentry{amplificateur}{
name={amplificateur}, plural={amplificateurs},
description={élément \textit{cis}-régulateur de l'expression d'un gène dont l'activité participe à l'initiation et/ou l'amplification de la transcription}}
 
\newglossaryentry{cis}{name={\textit{cis}},description={séquence d'ADN ou mécanisme de régulation dont les gènes cibles sont situés sur le même chromosome.}}

\newglossaryentry{condition}{name={condition biologique}, plural={conditions}, description={décrit l'état d'une cellule, peut refèrer à des types cellulaires, tissus ou organes distincts, des stades du développement, ou un environnement biochimique}}

\newglossaryentry{trans}{name={\textit{trans}},description={séquence d’ADN ou mécanisme de régulation agissant par le biais de produits (protéines ou ARNs) qui circulent dans le noyau des cellules et dont les gènes cibles ne sont pas nécessairement situés sur le même chromosome.}}

\newglossaryentry{transcriptome}{
name={transcriptome}, plural={transcriptomes},
description={ensemble des ARNs issus de la transcription du génome présents dans une cellule à un instant donné}}


\newglossaryentry{pleiotrope}{name={pléiotrope}, plural={pléiotropes}, user1={pleiotropie}, description={Décrit une séquence (gène ou élément \textit{cis}-régulateur) active dans plusieurs conditions biologiques et pouvant potentiellement agir sur plusieurs fonctions et caractères phénotypiques}}

% in vivo; in situ, in vitro

%\printglossary[type=\acronymtype]
%\printglossary

%\mtcaddchapter
%\mtcaddchapter